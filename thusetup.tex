% !TeX root = ../main.tex

% 论文基本信息配置

\thusetup{
  %******************************
  % 注意:
  %   1. 配置里面不要出现空行
  %   2. 不需要的配置信息可以删除
  %******************************
  %
  % 标题
  %   可使用“\\”命令手动控制换行
  %
  title  = {基于语义分析的整型缺陷检测与\\程序理解},
  title* = {Integer Defect Detection and Program Understanding Based on Semantic Analysis},
  %
  % 学位
  %   1. 学术型
  %      - 中文
  %        需注明所属的学科门类,例如:
  %        哲学、经济学、法学、教育学、文学、历史学、理学、工学、农学、医学、
  %        军事学、管理学、艺术学
  %      - 英文
  %        博士:Doctor of Philosophy
  %        硕士:
  %          哲学、文学、历史学、法学、教育学、艺术学门类,公共管理学科
  %          填写“Master of Arts“,其它填写“Master of Science”
  %   2. 专业型
  %      直接填写专业学位的名称,例如:
  %      教育博士、工程硕士等
  %      Doctor of Education, Master of Engineering
  %   3. 本科生不需要填写
  %
  degree-name  = {工学硕士},
  degree-name* = {Master of Science},
  %
  % 培养单位
  %   填写所属院系的全名
  %
  department = {软件学院},
  %
  % 学科
  %   1. 学术型学位
  %      获得一级学科授权的学科填写一级学科名称,其他填写二级学科名称
  %   2. 工程硕士
  %      工程领域名称
  %   3. 其他专业型学位
  %      不填写此项
  %   4. 本科生不需要填写
  %
  discipline  = {软件工程},
  discipline* = {Software Engineering},
  %
  % 姓名
  %
  author  = {李兀},
  author* = {Li Wu},
  %
  % 指导教师
  %   中文姓名和职称之间以英文逗号“,”分开,下同
  %
  supervisor  = {顾明, 教授},
  supervisor* = {Professor Gu Ming},
  %
  % 副指导教师
  %
  % associate-supervisor  = {周旻副研究员},
  % associate-supervisor* = {Associate Research Fellow Zhou Min},
  %
  % 联合指导教师
  %
  % joint-supervisor  = {某某某教授},
  % joint-supervisor* = {Professor Mou Moumou},
  %
  % 日期
  %   使用 ISO 格式;默认为当前时间
  %
  % date = {2019-07-07},
  %
  % 密级和年限
  %   秘密, 机密, 绝密
  %
  % secret-level = {秘密},
  % secret-year  = {10},
  %
  % 博士后专有部分
  %
  % clc                = {分类号},
  % udc                = {UDC},
  % id                 = {编号},
  % discipline-level-1 = {计算机科学与技术},  % 流动站(一级学科)名称
  % discipline-level-2 = {系统结构},          % 专业(二级学科)名称
  % start-date         = {2011-07-01},        % 研究工作起始时间
}

%% Put any packages you would like to use here

% 表格中支持跨行
\usepackage{multirow}

% 跨页表格
\usepackage{longtable}

% 固定宽度的表格
\usepackage{tabularx}

% 表格中的反斜线
\usepackage{diagbox}

% 确定浮动对象的位置,可以使用 H,强制将浮动对象放到这里(可能效果很差)
\usepackage{float}

% 浮动图形控制宏包。
% 允许上一个 section 的浮动图形出现在下一个 section 的开始部分
% 该宏包提供处理浮动对象的 \FloatBarrier 命令,使所有未处
% 理的浮动图形立即被处理。这三个宏包仅供参考,未必使用:
% \usepackage[below]{placeins}
% \usepackage{floatflt} % 图文混排用宏包
% \usepackage{rotating} % 图形和表格的控制旋转

% 定理类环境宏包
\usepackage[amsmath,thmmarks,hyperref]{ntheorem}
\newcommand\F{F}
\newcommand\Pf{Pf}
\DeclareMathOperator{\End}{End}
\DeclareMathOperator{\Frob}{Frob}

% 给自定义的宏后面自动加空白
% \usepackage{xspace}

% 借用 ltxdoc 里面的几个命令。
\def\cmd#1{\cs{\expandafter\cmd@to@cs\string#1}}
\def\cmd@to@cs#1#2{\char\number`#2\relax}
\DeclareRobustCommand\cs[1]{\texttt{\char`\\#1}}

\newcommand*{\meta}[1]{{%
  \ensuremath{\langle}\rmfamily\itshape#1\/\ensuremath{\rangle}}}
\providecommand\marg[1]{%
  {\ttfamily\char`\{}\meta{#1}{\ttfamily\char`\}}}
\providecommand\oarg[1]{%
  {\ttfamily[}\meta{#1}{\ttfamily]}}
\providecommand\parg[1]{%
  {\ttfamily(}\meta{#1}{\ttfamily)}}
\providecommand\pkg[1]{{\sffamily#1}}

% 定义所有的图片文件在 figures 子目录下
\graphicspath{{./figures/}}

% 数学命令
\input{math_commands.tex}

% 定义自己常用的东西
% \def\myname{薛瑞尼}

\usepackage{parcolumns}

% 代码
\usepackage{listings}
\usepackage{xcolor}
\renewcommand{\lstlistingname}{代码}% Listing -> 代码

\definecolor{dkgreen}{rgb}{0,0.6,0}
\definecolor{gray}{rgb}{0.5,0.5,0.5}
\definecolor{mauve}{rgb}{0.58,0,0.82}

\lstset{
	language=c++,
	frame=none,
	showstringspaces=false,
	columns=fixed,
	basicstyle={\ttfamily}, %\small
	numbers=left,
%	numberstyle=\tiny\color{gray},
	keywordstyle=\color{blue},
	commentstyle=\color{dkgreen},
	stringstyle=\color{mauve},
	breaklines=true,
	breakatwhitespace=true,
	tabsize=2,	
	otherkeywords={uint},
}

% 算法
\usepackage{algorithm}
\usepackage{algpseudocode}
%\usepackage{algorithmic}
%\floatname{algorithm}{算法}
%\renewcommand{\algorithmicrequire}{\textbf{输入:}}
%\renewcommand{\algorithmicensure}{\textbf{输出:}}

% 跨页算法
\makeatletter
\newenvironment{breakablealgorithm}
{% \begin{breakablealgorithm}
	\begin{center}
		\refstepcounter{algorithm}% New algorithm
		\hrule height.8pt depth0pt \kern2pt% \@fs@pre for \@fs@ruled
		\renewcommand{\caption}[2][\relax]{% Make a new \caption
			{\raggedright\textbf{\ALG@name~\thealgorithm} ##2\par}%
			\ifx\relax##1\relax % #1 is \relax
			\addcontentsline{loa}{algorithm}{\protect\numberline{\thealgorithm}##2}%
			\else % #1 is not \relax
			\addcontentsline{loa}{algorithm}{\protect\numberline{\thealgorithm}##1}%
			\fi
			\kern2pt\hrule\kern2pt
		}
	}{% \end{breakablealgorithm}
		\kern2pt\hrule\relax% \@fs@post for \@fs@ruled
	\end{center}
}
\makeatother

% hyperref 宏包在最后调用
\usepackage{hyperref}
