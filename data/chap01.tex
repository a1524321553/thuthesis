% !TeX root = ../main.tex

\chapter{引言}

 \section{研究背景与意义}
 
 突出整型缺陷会造成什么重大问题,解决这个问题十分重要。
 
 【这里需要重新找例子,更切题一点的】
 
 \section{国内外研究现状}
 
 主要围绕整型缺陷这边来延伸与展开。
 
 \subsection{整型缺陷检测技术与工具}
 
 介绍目前识别整型缺陷的常用技术与工具,阐述它们的优缺点。
 
 【这里可能需要额外做一些调研,然后列举在这上面】
 
 \subsection{抽象解释技术}
 
论证程序正确性的方法最朴素的便是穷举程序所有可能的输入,并通过执行得到结果来判断其是否符合预期,如果运行结果符合预期,那么程序自然是正确的。然而,这种方法只是一种理论上可能的方法,在实际中,我们面对的程序输入的取值范围往往非常大,甚至无法穷举。以C语言的函数举例,若该函数有一个int型参数,由于int类型的表示范围是$ \left[  -2147483648, +2147483647 \right] $,那么单单一个int型参数就有$ 2^{32}  $种取值,当输入的参数是字符串类型时,更是有无穷多中可能。因此,使用朴素的穷举来进行程序分析,其时间与空间代价在实际中是不可接受的。
 
 当前常用的测试技术便是采用了上述思想,只不过测试技术所选择的输入是总输入空间的子集,通过边界条件分析等方法得到相对较少的输入空间。该方法的优点是大幅减少测试输入,但也带来了程序运行路径覆盖率低、需要人工参与测试输入样例的设计等问题。
 
  相较于测试技术,抽象解释技术采用了不同的思路。抽象解释是一种对程序语义进行可靠抽象(近似)的通用理论\cite{cousot1977abstract}。与此同时,该理论为程序分析的设计与构建提供了一个通用的框架\cite{cousot1979systematic}。具体地,它是将程序语义进行不同程度的抽象,并将这种抽象及在其上的操作称为抽象域。通过将具体域中的值与抽象域中的值进行映射,从而将具体域中数量庞大甚至无穷大的取值域转化为抽象域中的有穷的取值域。并将具体域上的操作对应到抽象域上的操作,通过在抽象域上计算程序的抽象不动点来表达程序的抽象语义。
 
 单纯通过构建在抽象域上的与操作如迁移函数来进行建模有时并不能保证在程序的迭代分析中抽象域能快速到达不动点以获得抽象语义。因此在抽象分析中提供了加宽算子(widening),通过上近似理论来减少程序分析中的迭代次数,从而加速程序分析。由于上近似理论的可靠性,所有基于上近似抽象得到的性质,在源程序中必定成立。
 
 抽象解释的核心问题是抽象域的设计,而如上所述,抽象解释是对程序语义的不同程度的抽象,这也就意味着抽象域并不唯一确定,针对特定问题可以设计使用特定抽象域以达到程序分析的效果。目前为止,已经出现了数十种面向不同性质的抽象域,其中,具有代表性的抽象域包括区间抽象域、八边形抽象域、多面体抽象域等数值抽象域\cite{张健2019程序分析研究进展}。另一方面,在开源领域出现了众多抽象域库,如APRON\cite{jeannet2009apron}、ELINA\cite{singh2017practical}、PPL\cite{bagnara2006parma}等。
 
 抽象解释并不是一个已经研究成熟的课题,当下抽象解释仍然面临着很多挑战,主要包括两方面的内容:提高分析精度与拓展性。在提高分析精度方面,主要要解决的问题是基于加宽算子(widening)的不动点迭代运算的精度损失问题以及所设计的抽象域本身的表达能力具有局限性的问题。而在提高可拓展性方面,主要面临的问题是如何有效降低分析过程中抽象状态表示与计算的时空开销。
 
 【阐述与本文的关系】
 
 \subsection{常用抽象域与区间抽象域}
 
 具体展开抽象域的研究,首先介绍抽象域在静态分析方法中的角色与作用,随后剖析各个抽象域的优缺点,重点介绍区间抽象域,它能解决什么问题,为什么它比较好。
 
 【将上面的部分分一点儿到下面来】
 
 \subsection{总结}
 
 接上,阐述我们为什么要做区间抽象域,期望能达到什么样的一个目标,解决了传统区间抽象域的哪些痛点。
 
 \section{研究难点与挑战}
 
 难点大致在抽象域的设计方面:变迁规则、抽象方法(近似手段、合并操作等)
 
 \section{研究内容}
 
 这里搞一张图,到时候用这个说
 
 \begin{enumerate}
 	
 	\item 理论研究【暂未明确】
	 	\begin{enumerate}
	 		\item 基于程序解释的符号敏感的区间抽象域分析方法/基于线性空间的整型缺陷检测方法
	 		\item 【待讨论】基于二进制串的符号敏感的区间抽象域分析方法/基于环状区间的整型缺陷检测方法
	 		\item 基于区间分析的数值导向型缺陷分析组合方法
	 	\end{enumerate}
 	
 	\item 工具研发
 	
 \end{enumerate}
 
 \section{研究方案}
 
 这里同样补一张图,对应于上面的研究内容。不用像开题报告那样分小章节说,直接一段话即可。
 
 \section{论文贡献}
 
 最后补上。
 
 \section{论文组织结构}
 
 最后补上。