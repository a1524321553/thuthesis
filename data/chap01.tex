% !TeX root = ../main.tex

\chapter{引言}

 \section{研究背景与意义}
 
 突出整型缺陷会造成什么重大问题,解决这个问题十分重要。
 
 【这里需要重新找例子,更切题一点的】
 
 \section{国内外研究现状}
 
 主要围绕整型缺陷这边来延伸与展开。
 
 \subsection{整型缺陷检测技术与工具}
 
 介绍目前识别整型缺陷的常用技术与工具,阐述它们的优缺点。
 
 【这里可能需要额外做一些调研,然后列举在这上面】
 
 \subsection{抽象解释技术}
 
 具体落实到静态分析技术,介绍静态分析技术的原理与常用手段,如程序流自动机、CPA算法、抽象域等。
 
 介绍抽象解释技术,并为接下来章节所提到的技术做一定的铺垫。
 
 
 \subsection{常用抽象域与区间抽象域}
 
 具体展开抽象域的研究,首先介绍抽象域在静态分析方法中的角色与作用,随后剖析各个抽象域的优缺点,重点介绍区间抽象域,它能解决什么问题,为什么它比较好。
 
 \subsection{总结}
 
 接上,阐述我们为什么要做区间抽象域,期望能达到什么样的一个目标,解决了传统区间抽象域的哪些痛点。
 
 \section{研究难点与挑战}
 
 难点大致在抽象域的设计方面:变迁规则、抽象方法(近似手段、合并操作等)
 
 \section{研究内容}
 
 这里搞一张图,到时候用这个说
 
 \begin{enumerate}
 	
 	\item 理论研究【暂未明确】
	 	\begin{enumerate}
	 		\item 基于程序解释的符号敏感的区间抽象域分析方法/基于线性空间的整型缺陷检测方法
	 		\item 【待讨论】基于二进制串的符号敏感的区间抽象域分析方法/基于环状区间的整型缺陷检测方法
	 		\item 基于区间分析的数值导向型缺陷分析组合方法
	 	\end{enumerate}
 	
 	\item 工具研发
 	
 \end{enumerate}
 
 \section{研究方案}
 
 这里同样补一张图,对应于上面的研究内容。不用像开题报告那样分小章节说,直接一段话即可。
 
 \section{论文贡献}
 
 最后补上。
 
 \section{论文组织结构}
 
 最后补上。