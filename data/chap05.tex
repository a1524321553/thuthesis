% !TeX root = ../main.tex

\chapter{总结与展望}

\section{工作总结}

本文的工作主要包含如下两个方面的内容:

\begin{enumerate}[1)]
	\item 设计并实现基于线性区间的整型缺陷检测工具。本文首先借助一个
\end{enumerate}





\section{研究展望}


%软件的验证与确认在软件研发过程中一直是重点关注的内容。本文聚焦于帮助开发人员快速发现并定位软件中可能存在的数值导向型缺陷,以及
%
%减少软件验证中存在的困难,同时通过
%
%同时聚焦于帮助软件维护人员解决软件确认过程中程序理解困难的问题,从而加速软件确认工作的效率。
%
%提出了基于精确值流图的程序理解与需求确认方法。总体上,本文完成了以下工作内容:
%
%\begin{enumerate}[1)]
%	\item 设计并实现了线性区间理论域,它以区间的形式刻画了C语言整型变量的可能取值范围。主要解决的技术难点是线性区间理论域的设计。
%\end{enumerate}
%
%
%其核心是理论域的设计。本章先介绍了整体的区间设计,随后依次对各个抽象层面进行介绍,并结合控制流自动机,设计了整型变量的缺陷检测规则。最后,在基于Tsmart-V3静态分析框架上做了实现,并通过实验验证了其有效性。
%
%在\ref{sec:Integer}小节,我们介绍了扩展的整数RangeInteger,相比于整数,其具体定义了$ +\infty $和$ -\infty $的概念,同时,我们介绍了有$ -\infty, +\infty $参与的整数运算规则,为后续介绍区间抽象域奠定了基础。在\ref{sec:Range}小节,介绍了基于扩展整数的区间Range,它以单独区间的形式表示了一个整型变量的可能取值范围,并首次定义整数区间上的运算规则。
%
%随后,在\ref{sec:MultiRange}和\ref{sec:SignRange}小节,我们介绍了线性多区间MultiRange与位敏感的线性多区间SignRange,它们的提出分别解决了单独的Range区间精度不足以及无法描述符号性的问题。随后为了进一步提升多区间在区间长度较小时的精度,提出了基于拆分-合并思想的精度提升方法。
%
%最后,结合CPA,提出了基于区间算数的缺陷检测方法。该方法基于区间状态RangeState抽象域,它是CFA上程序状态节点的抽象。通过设计抽象域与变迁规则,可以得到整型变量在程序指令执行前后的取值区间,从而实现整型缺陷分析,并通过实验验证了有效性。
%
%在实验中,测试用例的误报主要原自基于for循环的检测。由于循环摘要的精度受制于循环不变式技术,为了进一步提升整型缺陷分析在循环中的精度,后续的工作重点将集中在循环不变式上。