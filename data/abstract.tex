% !TeX root = ../main.tex

% 中英文摘要和关键字

\begin{abstract}

软件的验证与确认是软件研发过程中不可或缺的一环。前者关注软件是否如定义般实现,后者关注软件是否符合预期的需求。在软件实现中,开发人员常不可避免的因疏忽引入缺陷。一方面,部分具有特定模式的缺陷因现有检测工具表示精度的问题无法被及时发现与确认,这将给软件验证工作带来隐患;另一方面,代码中的深层缺陷,尤其是与需求的匹配性,则需要基于对代码的深层理解来实现。在实际工作中,这一过程通常占用较多的时间。因此,提高开发人员对程序代码的理解效率成为一个重点问题。

本文针对上述问题,试图从程序的整型缺陷检测与整型变量关系分析两个方面入手,在一定范围内解决问题。本文的工作内容如下:
(1)针对当前线性区间抽象域由于表示能力不足而造成的程序分析精度损失的问题,提出了基于位敏感的区间代数的整型缺陷检测方法。具体地,本文依次设计了线性区间、线性多区间以及位敏感的线性多区间理论域与计算规则,能够较好的表示整型变量的可能取值范围;其次,为了得到程序的语义表示,本文设计了区间抽象域来描述程序的每个状态,并通过设计变迁规则,模拟程序状态的变迁,从而实现程序语义分析;最后通过设计缺陷检测规则实现程序的整型缺陷检测。
(2)为了解决软件确认过程中因程序理解造成的低效问题,提出基于值流图的整型变量关系分析与缺陷检测方法。具体地,介绍了精确值流图的定义与构造,并在此基础上设计了程序语义的抽取方法,包括抽象域的设计、变迁规则的设计和表达式分析算法的设计。使用算法,可以对程序中的整型变量在不同程序路径下的取值关系进行语义抽取,并生成模块摘要。(3)基于上述两个研究工作所提出的解决方案,在Tsmart静态分析框架中进行编码实现,并通过实验验证了其正确性与实用性。

经过实验,本文提出的基于区间代数的整型缺陷方法在Juliet测试集上可达到0\%的漏报率与低于3\%的漏报率;本文提出的基于值流图的整型变量关系分析与缺陷检测方法能够在选用的程序代码上正确的生成程序摘要,并可有效提升研发人员对程序代码的理解效率,具有良好的应用价值。
 
 \thusetup{
	keywords = {软件验证与确认, 缺陷检测, 程序理解, 静态分析},
}
\end{abstract}

\begin{abstract*}
	
Software verification and validation is an indispensable part of the software development process. The former focuses on whether the software is implemented as defined, while the latter focuses on whether the software meets the expected requirements. In software implementation, it is inevitable for developers to introduce defects due to negligence. On the one hand, some defects with specific patterns cannot be found and confirmed in time due to the representation accuracy of existing detection tools, which will bring hidden weakness to the software verification work; on the other hand, the deep defects in the code, especially the matching with the requirements, need to be realized based on the deep understanding of the code. In practice, this process usually takes more time. Therefore, it is a key problem to improve the efficiency of developers' understanding of program code.

In view of the above problems, this paper attempts to solve the problems in a certain range from the two aspects of program integer defect detection and integer variable relationship analysis. The work of this paper is as follows:
(1) Aiming at the problem of the loss of program analysis precision caused by the lack of representation ability in the current linear interval abstract domain, an integer defect detection method based on bitwidth sensitive interval algebra is proposed. Specifically, this paper designs the linear interval, linear interval and bitwidth sensitive linear interval theory domain and calculation rules in turn, which can better represent the possible value range of integer variables; secondly, in order to obtain the semantic representation of the program, this paper designs the interval abstract domain to describe each state of the program, and simulates the change of the program state by designing the transition rules Finally, the integer defect detection is realized by designing defect detection rules.
(2) In order to solve the problem of low efficiency caused by program understanding in the process of software validation, an integer variable relationship analysis and defect detection method based on value flow graph is proposed. Specifically, this paper introduces the definition and construction of the precise value flow graph, and designs the extraction method of program semantics, including the design of abstract domain, the design of transition rules and the design of expression analysis algorithm. Using the algorithm, we can extract the value relationship of integer variables in different program paths and generate module summary. (3) Based on the solutions proposed by the above two research works, the code is implemented in the Tsmart static analysis framework, and its correctness and practicability are verified by experiments.

Through experiments, the algorithm based on interval arithmetic can achieve 0\% false negative rate and less than 3\% false positive rate in Juliet test set; the algorithm based on value flow graph can generate program summary correctly in the selected program code, and can effectively improve the understanding efficiency of developers on the program code, which has good application value.



  \thusetup{
    keywords* = {Software verification and validation, defect detection, program understanding, static analysis},
  }
\end{abstract*}
